\chapter{Práctica Tkinter Robótica}

\section{Implementación de la interfaz gráfica con tkinter}

El codigo de este script puede encontrarse en el archivo \textbf{mainInterfaz.py}

En las siguientes figuras puede verse como es la apariencia de la interfaz gráfica con todas las ventanas tal y como se muestra en el guión de practicas:

\begin{figure}[H]
	\centering
	\includegraphics[width=0.8\textwidth]{mainInterfaz1.png}
	\caption{Interfaz gráfica recién iniciada}
\end{figure}
\begin{figure}[H]
	\centering
	\includegraphics[width=0.8\textwidth]{mainInterfaz2.png}
	\caption{Interfaz gráfica funcionando}
\end{figure}
\begin{figure}[H]
	\centering
	\includegraphics[width=0.8\textwidth]{mainInterfaz3.png}
	\caption{Botón Conectar con VREP}
\end{figure}
\begin{figure}[H]
	\centering
	\includegraphics[width=0.8\textwidth]{mainInterfaz4.png}
	\caption{Botón Conectar con VREP}
\end{figure}
\begin{figure}[H]
	\centering
	\includegraphics[width=0.8\textwidth]{mainInterfaz5.png}
	\caption{Botón Capturar}
\end{figure}
\begin{figure}[H]
	\centering
	\includegraphics[width=0.8\textwidth]{mainInterfaz6.png}
	\caption{Botón Capturar}
\end{figure}
\begin{figure}[H]
	\centering
	\includegraphics[width=0.8\textwidth]{mainInterfaz7.png}
	\caption{Botón Capturar}
\end{figure}
\begin{figure}[H]
	\centering
	\includegraphics[width=0.8\textwidth]{mainInterfaz9.png}
	\caption{Botón Detener y Desconectar VREP}
\end{figure}
\begin{figure}[H]
	\centering
	\includegraphics[width=0.8\textwidth]{mainInterfaz8.png}
	\caption{Botón Salir}
\end{figure}
\begin{figure}[H]
	\centering
	\includegraphics[width=0.8\textwidth]{mainInterfaz10.png}
	\caption{Botón Salir}
\end{figure}

\newpage

\section{Captura de los datos del laser 2D en diferentes situaciones}

El codigo de este script puede encontrarse en el archivo \textbf{capturar.py}

Los parámetros escogidos para la captación de datos son:
\begin{itemize}
	\item Iteraciones: 50
	\item Cerca: 0.5
	\item Media: 1.5
	\item Lejos: 2.5
\end{itemize}

A continuación, se pueden ver las diferentes escenas que se han usado para la captación de datos:

\begin{figure}[H]
	\centering
	\includegraphics[width=0.8\textwidth]{capturar1.png}
	\caption{Escena: escenaenPie.ttt}
\end{figure}
\begin{figure}[H]
	\centering
	\includegraphics[width=0.8\textwidth]{capturar2.png}
	\caption{Escena: escenasentado.ttt}
\end{figure}
\begin{figure}[H]
	\centering
	\includegraphics[width=0.8\textwidth]{capturar3.png}
	\caption{Escena: escenacilindroMenor.ttt}
\end{figure}
\begin{figure}[H]
	\centering
	\includegraphics[width=0.8\textwidth]{capturar4.png}
	\caption{Escena: escenacilindorMenorPared.ttt}
\end{figure}
\begin{figure}[H]
	\centering
	\includegraphics[width=0.8\textwidth]{capturar5.png}
	\caption{Escena: escenacilindroMayor.ttt}
\end{figure}
\begin{figure}[H]
	\centering
	\includegraphics[width=0.8\textwidth]{capturar6.png}
	\caption{Escena: escenacilindroMayorPared.ttt}
\end{figure}

Los objetos que pueden verse en las imagenes, se mueven justo delante del robot, a la distancia justa y se desplazan una distancia determinada por el número de iteraciones hasta llegar a la distancia máxima. Durante este desplazamiento, el modelo rota de manera que se puedan captar dos rotaciones completas.

Se puede notar, que tal y como se explicó en las sesiones de prácticas, se han introducido dos tipos de escenas para los casos negativos, una que no cuenta con paredes y otra que si. Posteriormente se analizarán las variaciones entre los resultados.

Los datos recogidos durante las pruebas se encuentran disponibles en las carpetas positivo1-6 y negativo1-6. Por defecto, se encuentran disponibles los datos para los que se han utilizado las escenas de casos negativos con paredes.
No obstante, para poder replicar los resultados de la predicción se han includo los archivos \textbf{DatosParedes.zip} y \textbf{DatosSinParedes.zip}, que contienen los datos recogidos con las escenas con paredes y sin paredes respectivamente. Para usar unos datos u otros, simplemente se deberán descomprimir estos archivos.

\newpage

\section{Agrupar los puntos x,y en clústeres}

El codigo de este script puede encontrarse en el archivo \textbf{agrupar.py}

Los parámetros escogidos son:
\begin{itemize}
	\item MinPuntos: 3 \\ He considerado que 3 puntos son los minimos necesarios para poder extraer profundidad de un cluster.
	\item MaxPuntos: 25 \\ He establecido el límite máximo lo suficientemente grande para captar los cilindros grandes al encontrarse cerca.
	\item UmbralDistancia: 0.05 \\ Un valor que ha aportado buenos resultados de manera experimental.
\end{itemize}


\begin{figure}[H]
	\centering
	\includegraphics[width=0.8\textwidth]{agruparParedes.png}
	\caption{Agrupación de los datos (con paredes en los negativos)}
\end{figure}
\begin{figure}[H]
	\centering
	\includegraphics[width=0.8\textwidth]{agruparSinParedes.png}
	\caption{Agrupación de los datos (sin paredes en los negativos)}
\end{figure}

\section{Convertir los clústeres en características \\ geométricas}

El codigo de este script puede encontrarse en el archivo \textbf{caracteristicas.py}

\section{Entrenar un clasificador binario utilizando Support Vector Machine (SVM) y Scikit-Learn}

El codigo de este script puede encontrarse en el archivo \textbf{clasificar.py}

\begin{figure}[H]
	\centering
	\includegraphics[width=0.8\textwidth]{SVCparedes.png}
	\caption{Resultados de SVC con los datos con paredes}
\end{figure}

\begin{figure}[H]
	\centering
	\includegraphics[width=0.8\textwidth]{SVCsinparedes.png}
	\caption{Resultados de SVC con los datos sin paredes}
\end{figure}

\section{Utilizar el clasificador con datos nuevos a partir del simulador}

El codigo de este script puede encontrarse en el archivo \textbf{predecir.py}

\begin{figure}[H]
	\centering
	\includegraphics[width=0.8\textwidth]{predecir.png}
	\caption{Escena: escenatest.ttt}
\end{figure}

\begin{figure}[H]
	\centering
	\includegraphics[width=0.8\textwidth]{lecturaLaser.jpg}
	\caption{Lectura del láser de robot}
\end{figure}

\subsection{Datos con paredes}
\begin{figure}[H]
	\centering
	\includegraphics[width=0.8\textwidth]{predecirParedes/terminal.png}
	\caption{Salida del programa}
\end{figure}

\begin{figure}[H]
	\centering
	\includegraphics[width=0.8\textwidth]{predecirParedes/prediccion.jpg}
	\caption{Predicción}
\end{figure}

\subsection{Datos sin paredes}
\begin{figure}[H]
	\centering
	\includegraphics[width=0.8\textwidth]{predecirSinParedes/terminal.png}
	\caption{Salida del programa}
\end{figure}

\begin{figure}[H]
	\centering
	\includegraphics[width=0.8\textwidth]{predecirSinParedes/prediccion.jpg}
	\caption{Predicción}
\end{figure}
